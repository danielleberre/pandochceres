
\documentclass[]{article}
\usepackage{helvet}
\usepackage[explicit]{titlesec}
\usepackage{titling}
\usepackage{array}
\renewcommand{\familydefault}{\sfdefault}
\usepackage{multicol}
\setlength{\columnsep}{1cm}

\usepackage{lmodern}
\usepackage{amssymb,amsmath}
\usepackage{ifxetex,ifluatex}
\usepackage{fixltx2e} % provides \textsubscript
\ifnum 0\ifxetex 1\fi\ifluatex 1\fi=0 % if pdftex
  \usepackage[T1]{fontenc}
  \usepackage[utf8]{inputenc}
\else % if luatex or xelatex
  \ifxetex
    \usepackage{mathspec}
  \else
    \usepackage{fontspec}
  \fi
  \defaultfontfeatures{Ligatures=TeX,Scale=MatchLowercase}
\fi
% use upquote if available, for straight quotes in verbatim environments
\IfFileExists{upquote.sty}{\usepackage{upquote}}{}
% use microtype if available
\IfFileExists{microtype.sty}{%
\usepackage{microtype}
\UseMicrotypeSet[protrusion]{basicmath} % disable protrusion for tt fonts
}{}
\usepackage[top=2cm, bottom=1.5cm, left=1cm, right=1cm]{geometry}
\usepackage[unicode=true]{hyperref}
\hypersetup{
            pdftitle={Unité de recherche},
            pdfborder={0 0 0},
            breaklinks=true}
\urlstyle{same}  % don't use monospace font for urls
\IfFileExists{parskip.sty}{%
\usepackage{parskip}
}{% else
\setlength{\parindent}{0pt}
\setlength{\parskip}{6pt plus 2pt minus 1pt}
}
\setlength{\emergencystretch}{3em}  % prevent overfull lines
\providecommand{\tightlist}{%
  \setlength{\itemsep}{0pt}\setlength{\parskip}{0pt}}
\setcounter{secnumdepth}{5}
% Redefines (sub)paragraphs to behave more like sections
\ifx\paragraph\undefined\else
\let\oldparagraph\paragraph
\renewcommand{\paragraph}[1]{\oldparagraph{#1}\mbox{}}
\fi
\ifx\subparagraph\undefined\else
\let\oldsubparagraph\subparagraph
\renewcommand{\subparagraph}[1]{\oldsubparagraph{#1}\mbox{}}
\fi

% set default figure placement to htbp
\makeatletter
\def\fps@figure{htbp}
\makeatother

\usepackage[table]{xcolor}
\usepackage{pgf}
\usepackage{tikz}
\usepackage{sectsty}
\usepackage{tcolorbox}
\definecolor{hceresred}{HTML}{ED145B}
\definecolor{hcerespurple}{HTML}{5C2D91}
\definecolor{hceresgreen}{HTML}{76923C}
\sectionfont{\color{hceresred}\MakeUppercase} 
\subsectionfont{\color{hcerespurple}}
\subsubsectionfont{\color{hcerespurple}}
\usepackage{fancyhdr}
\pagestyle{fancy}
\setlength{\headheight}{2cm}
\setlength{\voffset}{0cm}
\setlength{\textheight}{22cm}
\setlength{\footskip}{1.5cm}
\renewcommand{\headrulewidth}{0pt}
\lhead{Document d'autoévaluation des unités de recherche}
\chead{}
\rhead{\raisebox{-.5\height}{\pgfimage[width=1.5cm]{hceres}}}
\lfoot{Campagne d’évaluation  2024-2025 – Vague E\hspace{1cm} Département d’évaluation de la recherche}
\cfoot{}
\rfoot{\thepage}
\tcbsetforeverylayer{colframe=hceresred,colback=white,sharp corners,boxrule=1pt,boxsep=5mm}
\newcommand{\instructions}[1]{{\em \color{hceresgreen}#1}}
\title{Unité de recherche}
\pagestyle{fancy}
\begin{document}

\begin{titlepage}
\tikz[remember picture,overlay] \node[inner sep=0pt] at ([xshift=8.5cm,yshift=-3cm]current page.north west){\raisebox{-.5\height}{\includegraphics[width=4cm]{hceres}}  \Large\bf\color{hceresred} Département d’évaluation de la recherche};

\tikz[remember picture,overlay] \node[inner sep=0pt] at ([xshift=-8.2cm,yshift=-3cm]current page.east){\includegraphics[width=16cm]{hceresfront}};

\begin{tcolorbox}[standard jigsaw, opacityback=0, opacityframe=0,boxsep=.7cm,left skip=4cm, right skip=-4cm, before skip=5cm]
{\color{hcerespurple}\LARGE DOCUMENT\\
D'AUTOÉVALUATION DES\\
UNITÉS DE RECHERCHE}


\vskip 2cm
{\raggedright\color{hceresred}\bf\large CAMPAGNE D'EVALUATION 2024-2025
}\\
{\raggedright\color{hceresred}\large Vague E
}
\par\vskip 2cm
{\raggedright\color{hcerespurple}\large Septembre 2023}

\par
\end{tcolorbox}

\date{Septembre 2023}
\end{titlepage}


\instructions{L’unité rédige ce document en complétant les rubriques ci-dessous et en suivant le plan indiqué. Les parties en vert dans le texte sont des repères pour l’autoévaluation. Elles doivent être supprimées dans le document rédigé.}

\section{Informations générales pour le contrat en cours}

\subsection{Identification de l'unité}

Nom de l'unité :

Acronyme~:

Label et numéro~:

Domaine scientifique principal~:

Panels scientifiques par ordre décroissant de pertinence~:

\begin{itemize}
\tightlist
\item
  Panel 1\\
  Choisissez un élément.
\item
  Panel 2\\
  Choisissez un élément.
\item
  Panel 3\\
  Choisissez un élément.
\item
  Panel 4\\
  Choisissez un élément.
\end{itemize}

Équipe de direction~:

\instructions{On précisera le rôle de chaque membre de l’équipe de direction.}

Liste des tutelles de l'unité de recherche :

Écoles doctorales de rattachement~:

\instructions{Nom complet.}

\subsection{Présentation de l'unité}

Historique, localisation de l'unité~:

Organisation de l'unité :

Équipes, plateformes, services communs, etc~:

Effectif de l'unité et de ses éventuelles équipes au 31/12/2023~:

Thématiques scientifiques (par équipe le cas échéant)~:

\instructions{Les nommer, la présentation des thématiques se situant dans l’item 3.}

\subsection{Les thématiques scientifiques et leurs enjeux}

\instructions{L’unité de recherche est invitée à présenter les thématiques de recherche qu’elle a abordées au cours de la période de référence, en les situant dans le contexte international et {\bf en s’appuyant sur les faits marquants qu’elle considère comme des avancées scientifiques remarquables}. Ces faits scientifiques marquants seront également détaillés dans la réponse à la référence 1 du domaine 3 du référentiel de l’autoévaluation. Ils répondent à des enjeux scientifiques, technologiques, culturels, économiques ou sociétaux. Le cas échéant, cet exposé des thématiques scientifiques pourra prendre en compte la structuration de l’unité de recherche en équipes qui pourront présenter leurs propres thématiques. Ce texte pourra être substantiel tout en restant d’une longueur raisonnable et adapté à la taille de l’unité de recherche. Il trouvera un écho dans le profil d’activités défini dans le paragraphe 4 de ce chapitre ainsi que dans le portfolio, objet du deuxième chapitre de ce document d’autoévaluation. Les éléments principaux de ce paragraphe seront brièvement repris, au titre de contexte, dans le dernier chapitre consacré à la trajectoire de l’unité de recherche.}

\subsection{Profil d'activités liées à la recherche}

\instructions{La définition du profil permet de caractériser, selon sept grandes catégories (classées ici par ordre alphabétique), l’ensemble des activités conduites par le collectif de recherche. Le profil d’activités est décliné à l’échelle de l’unité et, le cas échéant, de ses équipes.}

\begin{tabular}{ |p{16cm}|r|  }
\hline
\rowcolor{gray!40}
Activités (Répartir 100 points sur ces 7 items) &  Points   \\
\hline
Administration et animation de la recherche : pilotage de la recherche (VP, direction d’institut, DAS, par exemple), participation à des instances d’évaluation (CNU, CoNRS, CSS, Hcéres, par exemple), responsabilité de dispositifs Idex ou Isite, direction de projets (ANR, Horizon Europe, ERC, CPER, PIA, France 2030, par exemple), responsabilités éditoriales dans des revues ou collections nationales et internationales. & 0 \\
\hline
Aide aux politiques publiques et expertise technique : pouvoirs publics aux niveaux européen, national et régional, entreprises, instances internationales comme FAO, OMS, etc. & 0 \\
\hline
Contribution à l’adossement d’enseignements innovants à la recherche : EUR, SFRI, etc. & 0 \\
Dissémination de la recherche : partage de connaissances avec le grand public, médiation scientifique, interface sciences et société. & 0 \\
\hline
Recherche et encadrement de la recherche. & 0 \\
\hline
Valorisation, transfert, innovation. & 0 \\
\hline
Autres activités. (à préciser en une ligne maximum). & 0 \\
\hline
\end{tabular}

\subsection{Environnement de recherche}

\instructions{L’unité présente de façon synthétique les structures de recherche et de valorisation dans lesquelles elle est impliquée, à l’échelle de l’établissement ou du site :
\begin{itemize}
\item liens avec des fédérations de recherche, des plateformes, une MSH, un OSU, etc. ;
\item contribution à un champ de recherche (campus, institut, pôle, secteur...) et description de celui-ci ;
\item implication dans le continuum entre laboratoires de recherche et structures de soins ; 
\item implication dans un dispositif créé par le PIA du type Idex, Isite, Labex, Équipex, PEPR, EUR, IHU, etc. ;
\item inscription dans des clusters régionaux ;
\item participation à des structures de valorisation, de transfert et de recherche partenariale (incubateurs, SATT, IRT, ITE, instituts Carnot, etc.) ;
\end{itemize}
}

\subsection{Prise en compte des recommandations du précédent rapport}

\instructions{L’unité présente de façon synthétique les actions entreprises pour mettre en œuvre les recommandations de la précédente évaluation à l’échelle de l’unité et de ses équipes. Elle en évalue les résultats.}

\section{Introduction du portfolio}

\instructions{Le portfolio est le support de l’évaluation qualitative des activités de l’unité. Il comprend un ensemble d’éléments que l’unité juge représentatifs de ses activités, de ses missions et de son environnement de recherche.\\

{\bf Le portfolio fait l’objet d’une introduction qui décrit sa composition et qui justifie les choix opérés dans sa composition}. L’exposé veille à installer un dialogue entre la définition du profil d’activités et les éléments sélectionnés pour la constitution du portfolio. Cette introduction n’excède pas la limite de 3 500 caractères (espaces comprises) pour une unité mono-équipe et 7 000 caractères (espaces comprises) pour une unité pluri-équipe. Cette introduction fait l’objet de ce chapitre.\\

{\bf Le portfolio en lui-même (l’ensemble des documents sélectionnés par l’unité) fera l’objet d’un dossier zip contenant les éléments le constituant}. Ce fichier zip sera déposé en annexe. Si ce fichier zip devait excéder 50 Mo, l’unité est invitée à créer un lien de téléchargement et à l’indiquer à la fin de ce chapitre 2.\\

{\bf Le nombre total d’éléments du portfolio doit tenir compte de la taille et de la structuration de l’unité de recherche. Il doit également rester dans une limite raisonnable} pour que le comité d’experts puisse s’en saisir de façon approfondie. Nous proposons, à titre indicatif, le cadre suivant :

Pour une unité mono-équipe, il s’élève à :
\begin{itemize}
\item unité de petite taille (moins de 19 permanents) : cinq éléments dont au minimum deux publications ;
\item unité de taille moyenne (entre 20 et 39 permanents) : huit éléments dont au minimum quatre publications ;
\item unité de grande taille (40 personnes ou plus) : onze éléments dont au minimum cinq publications ;
\end{itemize}

Pour une unité pluri-équipe, il s’élève par équipe à :
\begin{itemize}
\item équipe de très grande taille (supérieure à 20 permanents) : maximum sept éléments dont au minimum trois publications ;
\item équipe de grande taille (entre 10 et 19 permanents) : maximum cinq éléments dont au minimum deux publications ;
\item équipe de taille moyenne (entre 5 et 9 permanents) : maximum quatre éléments dont deux publications ;
\item équipe de petite taille (moins de 4 permanents) : maximum trois éléments dont une publication ;
\end{itemize}

L’unité pourra répartir ces éléments entre des productions relatives à chaque équipe et des productions à l’échelle de l’unité.

Pour les unités de recherche ayant plus de 15 équipes, la dimension du portfolio fera l’objet d’un échange avec le conseiller scientifique en charge de l’unité.
Le portfolio peut rassembler les éléments suivants :
\begin{itemize}
\item  des productions représentatives du positionnement scientifique de l’unité (front de connaissance, positionnement théorique, innovation méthodologique…) attestant notamment de sa reconnaissance aux niveaux national, européen et international (articles, ouvrages, créations artistiques, par exemple) ;
\item  des éléments soulignant l’implication de l’unité dans les activités d’encadrement et de formation (initiale ou à destination du monde professionnel) et témoignant des apports de l’activité scientifique de l’unité à la spécialisation de l’offre de formation de l’établissement (implication dans des projets EUR, d’universités européennes ou d’alliances pour l’innovation, conception de formations à destination de secteurs professionnels spécifiques, par exemple) ;
\item  des éléments présentant des dynamiques d’innovation sociale (co-production de recherche avec des acteurs non-académiques, collaboration de recherche avec des panels citoyens, par exemple) ;
\item  des éléments illustrant des actions de valorisation, de transfert (actions de coopération avec les collectivités territoriales, actions en matière d’aides aux politiques publiques, participation à des actions de repérage technologique et autres partenariats public-privé, etc.) et des contributions au développement socio-économique et culturel (note descriptive sur un contrat de R\&D significatif, sur le contexte de création d’une start-up, par exemple) ;
\item  des éléments soulignant des activités de dissémination de la recherche (mise en place d’évènements à destination du grand public, production de documents audio-visuels, podcasts, ouvrages, expertises auprès d’acteurs du monde social, économique, culturel, politique, etc.) ;
\item  tout autre élément que l’unité jugera pertinent pour apprécier des aspects singuliers de son profil d’activités.
\end{itemize}
}

\section{Autoévaluation du bilan}

\instructions{Selon les domaines d’évaluation, l’unité appuie son argumentation sur :
\begin{itemize}
\item les données fournies dans le tableur « Données de caractérisation et de production » ;
\item les éléments sélectionnés pour la constitution du portfolio ;
\item l’exposé de ses thématiques scientifiques du paragraphe 3 du chapitre 1 ;
\item des données fournies en annexe, le cas échéant.
\end{itemize}

Pour les unités pluri-équipes : on commence par décliner les quatre domaines à l’échelle de l’unité puis, pour chacune des équipes, on choisit parmi les domaines les références jugées pertinentes pour l’équipe. S’il n’est ni opportun ni nécessaire d’aborder pour chaque équipe toutes les références, celles relatives à la production, à l’attractivité et à l’inscription dans la société doivent être privilégiées en reprenant cette séquence.
Dans l’éventualité où toutes les références devraient être abordées, on veille à respecter l’ordre de présentation.
}

\subsection{Autoévaluation de l'unité}

\subsubsection*{Domaine 1. Profil, ressources et organisation de
l'unité}
\addcontentsline{toc}{subsubsection}{Domaine 1. Profil, ressources et
organisation de l'unité}

\instructions{Ce domaine se décline en trois références : adéquation de la politique de recherche mise en œuvre par l’unité à son potentiel humain ; moyens financiers et logistiques mobilisés ; pratiques responsables en matière de ressources humaines, de sécurité, et d’environnement.}

\paragraph*{Référence 1. L'unité s'est assigné des objectifs
scientifiques pertinents.}
\addcontentsline{toc}{paragraph}{Référence 1. L'unité s'est assigné des
objectifs scientifiques pertinents.}

\instructions{L’unité exprime sa vision de son environnement de recherche et de ses acteurs. Elle montre en particulier comment elle tient compte de la politique de ses tutelles en matière de recherche et de valorisation. Elle décrit sa stratégie scientifique et présente comment elle associe l’ensemble de ses personnels à l’élaboration de sa politique de recherche et de valorisation.\\

L’unité analyse les impacts scientifiques, économiques, culturels et sociétaux de la politique qu’elle conduit et elle décrit comment elle les prend en considération.}

\paragraph*{Référence 2. L'unité dispose de ressources adaptées à son
profil d'activités et à son environnement de recherche et les mobilise.}
\addcontentsline{toc}{paragraph}{Référence 2. L'unité dispose de
ressources adaptées à son profil d'activités et à son environnement de
recherche et les mobilise.}

\instructions{L’unité présente les ressources financières dont elle dispose de façon récurrente et celles qu’elle est capable de mobiliser, au-delà de la dotation allouée par ses tutelles. Elle décrit sa politique de mutualisation d’une partie de ses ressources pour favoriser l’émergence de thématiques novatrices et pour soutenir des activités collectives de recherche.\\

L’unité expose sa politique en matière de locaux et d’infrastructures scientifiques ou de ressources documentaires. Elle montre comment celle-ci est adaptée à ses objectifs scientifiques.}

\paragraph*{Référence 3. Les pratiques de l'unité sont conformes aux
règles et aux directives définies par ses tutelles en matière de gestion
des ressources humaines, de sécurité, d'environnement, de protocoles
éthiques et de protection des données ainsi que du patrimoine
scientifique.}
\addcontentsline{toc}{paragraph}{Référence 3. Les pratiques de l'unité
sont conformes aux règles et aux directives définies par ses tutelles en
matière de gestion des ressources humaines, de sécurité,
d'environnement, de protocoles éthiques et de protection des données
ainsi que du patrimoine scientifique.}

\instructions{L’unité définit sa politique de ressources humaines. Elle décrit en particulier de quelle manière sa gestion des ressources humaines est respectueuse de la parité et non discriminatoire en matière de formation, de mobilité interne et d’évolution des carrières de ses personnels. Elle montre qu’elle est attentive aux conditions de travail de ses personnels, à leur santé, à leur sécurité et à la prévention des risques psycho-sociaux. En particulier, elle précise les mesures prises en matière de lutte contre les violences sexistes et sexuelles, et contre les discriminations.\\

L'unité décrit toutes les procédures mises en place pour protéger son patrimoine scientifique et ses systèmes informatiques.\\

L’unité indique les dispositions qu’elle applique pour prévenir les risques environnementaux résultant de son activité et pour poursuivre des objectifs de développement durable. L’unité précise si elle est dotée d’une charte de développement durable inscrite dans son règlement intérieur. En particulier, elle montre comment elle prend en compte les critères de développement durable dans la définition des actions de recherche et des expérimentations. Elle détaille sa politique en matière de gestion des missions et des déplacements des personnels, et de gestion des déchets, des consommables et des rebuts. Elle décrit les mesures de sensibilisation mises en place pour les étudiants accueillis. Elle indique comment elle évalue ses bonnes pratiques en matière d’empreinte environnementale.\\

L’unité décrit son plan de continuité d'activité et comment elle anticipe des situations d'urgence.}

\paragraph*{Synthèse de l'autoévaluation}
\addcontentsline{toc}{paragraph}{Synthèse de l'autoévaluation}

\instructions{L’unité évalue ses forces et faiblesses au regard des références de ce domaine d’évaluation.}

\subsubsection*{Domaine 2. Attractivité}
\addcontentsline{toc}{subsubsection}{Domaine 2. Attractivité}

\instructions{Ce domaine se décline en quatre références qui portent respectivement sur le rayonnement scientifique des membres de l’unité, sur la qualité de sa politique d’encadrement et d’accueil en lien avec les écoles doctorales, sur sa capacité à obtenir des financements (appels à projets compétitifs), et sur la qualité de ses équipements et de leur gestion.}

\paragraph*{Référence 1. L'unité est attractive par son rayonnement
scientifique et s'insère dans l'espace européen de la recherche.}
\addcontentsline{toc}{paragraph}{Référence 1. L'unité est attractive par
son rayonnement scientifique et s'insère dans l'espace européen de la
recherche.}

\instructions{L’unité expose les actions qu’elle met en œuvre pour développer son rayonnement scientifique. Elle illustre ses résultats en la matière par des faits marquants : invitations des membres de l’unité dans des congrès, organisation de manifestations scientifiques, responsabilités éditoriales, participations à des instances de pilotage de la recherche, membres d’institutions, lauréats de prix, etc.}

\paragraph*{Référence 2. L'unité est attractive par la qualité de sa
politique d'accompagnement des personnels.}
\addcontentsline{toc}{paragraph}{Référence 2. L'unité est attractive par
la qualité de sa politique d'accompagnement des personnels.}

\instructions{L’unité présente sa politique d’accueil des nouveaux personnels. Elle mentionne les modalités d’accueil et d’intégration au sein des recherches de l’unité des chercheurs aussi bien débutants (de niveaux doctorat et post-doctorat) que confirmés (EC et C). Elle présente les résultats de cette politique. Elle expose l’accompagnement mis en place pour les personnels d’appui à la recherche.\\

L’unité souligne sa capacité à accueillir des chercheurs invités.\\

L’unité décrit la mise en œuvre de la stratégie de ses tutelles en matière d’intégrité scientifique et de science ouverte.}

\paragraph*{Référence 3. L'unité est attractive par la reconnaissance de
ses succès à des appels à projets compétitifs.}
\addcontentsline{toc}{paragraph}{Référence 3. L'unité est attractive par
la reconnaissance de ses succès à des appels à projets compétitifs.}

\instructions{L’unité décrit sa politique en matière de réponse à des appels à projets aussi bien internationaux que nationaux et locaux. Elle en présente les résultats.
Elle mentionne comment elle finance sur ses ressources propres des contrats doctoraux et post-doctoraux, des contrats d’ingénieur et de technicien, des chaires, des équipements.\\

L’unité expose son implication, à différents niveaux, dans des dispositifs et des projets financés par les programmes d’investissements nationaux (PIA, CPER, par exemple), et les bénéfices qu’elle en retire.}

\paragraph*{Référence 4. L'unité est attractive par la qualité de ses
équipements et de ses compétences techniques.}
\addcontentsline{toc}{paragraph}{Référence 4. L'unité est attractive par
la qualité de ses équipements et de ses compétences techniques.}

\instructions{L’unité indique l’ensemble de ses plateformes, de ses équipements, de ses démonstrateurs de pointe. Elle détaille sa stratégie de développement, de maintenance et de jouvence ainsi que d’ouverture à des tiers, de ses dispositifs. Elle explicite comment elle accède aux outils mis en place par ses tutelles pour acquérir et entretenir les équipements lourds.\\

Elle décrit et analyse la constitution de l’équipe technique et administrative engagée dans la gestion de ces équipements.}

\paragraph*{Synthèse de l'autoévaluation}
\addcontentsline{toc}{paragraph}{Synthèse de l'autoévaluation}

\instructions{L’unité évalue ses forces et faiblesses au regard des références de ce domaine d’évaluation.}

\subsubsection*{Domaine 3. Production scientifique}
\addcontentsline{toc}{subsubsection}{Domaine 3. Production scientifique}

\instructions{Ce domaine se décline en trois références qui portent respectivement sur les aspects qualitatifs, quantitatifs et éthiques de la production scientifique, et des résultats des recherches.}

\paragraph*{Référence 1. La production scientifique de l'unité satisfait
à des critères de qualité.}
\addcontentsline{toc}{paragraph}{Référence 1. La production scientifique
de l'unité satisfait à des critères de qualité.}

\instructions{L’unité analyse sa production scientifique. Elle s’appuie en particulier sur le portfolio et sur la liste de sa production pour montrer en quoi celle-ci repose sur des fondements théoriques et méthodologiques solides, qu’elle est originale, qu’elle présente un apport à la connaissance et qu’elle traduit un positionnement national et international des recherches menées par l’unité. 

Dans ce paragraphe, les principaux résultats scientifiques de l’unité seront repris du paragraphe « Les thématiques scientifiques et leurs enjeux » du 1er chapitre de ce document. Au cœur de l’approche qualitative de l’évaluation de la recherche de l’unité, ces faits scientifiques marquants (découvertes, inventions, avancées méthodologiques, nouveaux concepts, ruptures, etc.) seront détaillés et ils pourront faire l’objet d’un développement substantiel.}

\paragraph*{Référence 2. La production scientifique de l'unité est
proportionnée à son potentiel de recherche et correctement répartie
entre ses personnels.}
\addcontentsline{toc}{paragraph}{Référence 2. La production scientifique
de l'unité est proportionnée à son potentiel de recherche et
correctement répartie entre ses personnels.}

\instructions{L’unité présente sa stratégie interne de diffusion des connaissances. Elle analyse les éventuels déséquilibres de production entre ses différentes équipes. Elle décrit et analyse tout particulièrement la production des personnels chercheurs débutants. Elle mentionne les dispositifs mis en œuvre pour accompagner les personnels les moins actifs ou pour accompagner, sur ce point, les personnels chercheurs de niveaux doctorat et post-doctorat. Elle souligne l’apport des personnels d’appui à la recherche.}

\paragraph*{Référence 3. La production scientifique de l'unité respecte
les principes de l'intégrité scientifique, de l'éthique et de la science
ouverte. Elle est conforme aux directives applicables dans ce domaine.}
\addcontentsline{toc}{paragraph}{Référence 3. La production scientifique
de l'unité respecte les principes de l'intégrité scientifique, de
l'éthique et de la science ouverte. Elle est conforme aux directives
applicables dans ce domaine.}

\instructions{L’unité précise les moyens mis en œuvre pour garantir la traçabilité et, le cas échéant, la reproductibilité de ses résultats (carnets de laboratoires, logiciels anti-plagiat, procédures internes d’examen – dont de relecture – par les pairs, procédures d’archivage des données et des codes sources, etc.). Elle décrit les moyens par lesquels elle accompagne ses personnels dans le choix de supports appropriés de diffusion (pour éviter, par exemple, les conférences et revues dites « prédatrices ») et pour une juste prise en compte des contributions (en particulier dans les co-signatures).\\

L’unité indique les dispositions mises en place pour que sa production scientifique soit le résultat de recherches respectant la personne humaine, la vie animale. 
L’unité définit sa politique en matière de science ouverte.}

\paragraph*{Synthèse de l'autoévaluation}
\addcontentsline{toc}{paragraph}{Synthèse de l'autoévaluation}

\instructions{L’unité évalue ses forces et faiblesses au regard des références de ce domaine d’évaluation.}

\subsubsection*{Domaine 4. Inscription des activités de recherche dans
la société}
\addcontentsline{toc}{subsubsection}{Domaine 4. Inscription des
activités de recherche dans la société}

\instructions{Dans ce domaine, le mot « société » est entendu au sens large. L'inscription de l’activité de l’unité de recherche dans la société peut concerner l’économie, la santé, la culture, l’environnement, etc. Le domaine se décline en trois références, qui portent respectivement sur les interactions de l’unité avec les acteurs du monde non-académique, les produits de sa recherche à destination des acteurs socio-économiques et culturels et ses interventions dans la sphère publique.}

\paragraph*{Référence 1. L'unité se distingue par la qualité et la
quantité de ses interactions avec le monde non-académique.}
\addcontentsline{toc}{paragraph}{Référence 1. L'unité se distingue par
la qualité et la quantité de ses interactions avec le monde
non-académique.}

\instructions{L’unité est invitée à analyser ses partenariats avec les acteurs du monde culturel, économique et social et elle précise les modes de collaboration (conventions, contrats, etc.). Elle décrit l’ampleur de son activité avec le monde non-académique, par exemple au travers de mutualisation ou de convention d’accueil de personnels, de financement de doctorats (CIFRE, thèses financées par des contrats, etc.), de financement de ses activités de recherche, d’animation de formations continues ou d’activités de science participative ou collaborative.\\

L’unité indique comment elle se saisit de sujets à valeur scientifique, technologique, sociale et culturelle, en cohérence avec sa politique de recherche. Elle souligne comment ses partenariats permettent de relever des défis environnementaux, sociétaux ou technologiques.
}

\paragraph*{Référence 2. L'unité développe des produits à destination du
monde culturel, économique et social.}
\addcontentsline{toc}{paragraph}{Référence 2. L'unité développe des
produits à destination du monde culturel, économique et social.}

\instructions{L’unité présente sa politique de valorisation et les résultats obtenus en matière de développement de produits à destination du monde économique (brevets, licences, accompagnement de création d’entreprises, expertises, participation à la rédaction de normes, etc.).\\

L’unité décrit son activité de diffusion de ses résultats auprès des acteurs du monde social, économique et culturel.}

\paragraph*{Référence 3. L'unité partage ses connaissances avec le grand
public et intervient dans des débats de société.}
\addcontentsline{toc}{paragraph}{Référence 3. L'unité partage ses
connaissances avec le grand public et intervient dans des débats de
société.}

\instructions{L’unité expose et analyse sa politique de partage des connaissances avec le grand public et en particulier avec les populations scolaires. Elle présente les dispositions prises pour encourager la prise de parole de ses personnels dans l’espace public et pour que celle-ci se fasse dans le respect de l’intégrité scientifique et de la déontologie.}

\paragraph*{Synthèse de l'autoévaluation}
\addcontentsline{toc}{paragraph}{Synthèse de l'autoévaluation}

\instructions{L’unité évalue ses forces et faiblesses au regard des références de ce domaine d’évaluation.}

\subsection{Autoévaluation des équipes (dans le cas des unités
pluri-équipes)}

\instructions{Pour chacune des équipes, on choisit parmi les domaines les références jugées pertinentes pour l’équipe. S’il n’est ni opportun ni nécessaire d’aborder toutes les références, celles relatives à la production, à l’attractivité et à l’inscription dans la société doivent être privilégiées en reprenant cette séquence.
Dans l’éventualité où toutes les références devraient être abordées, on veille à respecter l’ordre de présentation.}

\section{Trajectoire de l'unité}

\instructions{La trajectoire est entendue selon deux dimensions : la dynamique et l’ambition de recherche, d’une part, l’organisation et la vie du laboratoire, d’autre part. Elle est décrite à l’échelle de l’unité et peut être ensuite déclinée à celle des équipes.
L’unité est invitée à décrire, de façon très synthétique, son historique scientifique de long terme et à rappeler les objectifs qu’elle s’était assignés lors de la précédente évaluation, la stratégie qu’elle avait mise en place, et les défis qu’elle comptait relever. Ces éléments de caractérisation scientifique permettent d’opérer une analyse critique, de confronter les réalisations aux objectifs initiaux, de discuter des réussites et des échecs. L’unité souligne les réorientations qu’elle a mises en œuvre.\\

L’unité précise comment elle s’inscrit aujourd’hui dans les champs de ses diverses interventions (scientifique, expertise, valorisation, formation, dissémination, etc.), aux niveaux national et international, en s’appuyant sur une analyse de l’état de l’art.\\

L’unité décrit sa projection scientifique sur la base de son autoévaluation, de ses acquis de recherche et des nouveaux enjeux de recherche identifiés. En se plaçant dans la perspective de son projet scientifique à cinq ans, l’unité présente sa vision prospective de l’évolution de son domaine scientifique, sa contribution aux questionnements en cours et le positionnement du projet dans le champ scientifique national ou international. Elle indique ses points d’appui, les points à améliorer et les possibilités offertes par son environnement. Elle précise les risques liés à cet environnement. Elle présente comment elle soutient l’émergence de nouvelles thématiques, les sujets de recherche à risque ou les disciplines rares.\\

L’unité expose, dans une vision prospective, sa stratégie partenariale avec le monde académique (aux échelles locale, nationale, européenne et internationale) et le monde socio-économique et culturel. L’unité est également invitée à montrer comment son projet s’intègre dans la stratégie des établissements tutelles et dans la stratégie du site universitaire.\\

L’unité justifie la mise en cohérence de sa stratégie de recherche avec ses moyens et son organisation : comment son organisation et ses évolutions ont servi ses objectifs scientifiques et comment sa future organisation et ses demandes de moyens répondront à ses ambitions. L’unité précise, dans ce paragraphe, les effectifs, les moyens à mobiliser et le mode de structuration (organisation,  positionnement  et contribution des équipes, synergies entre les équipes, plateformes) pour accompagner ses orientations, ses objectifs scientifiques et ses choix stratégiques. Elle présente un plan d’actions portant sur les nouveaux enjeux des laboratoires : science et société, science ouverte, impact environnemental des activités de l'unité, parité de genre, intégrité scientifique par exemple.
}

\end{document}
